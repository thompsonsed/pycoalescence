
\begin{DoxyRefList}
\item[\label{bug__bug000002}%
\hypertarget{bug__bug000002}{}%
Member \hyperlink{class_community_ac9433a0c34911ec1128b5459f61488fc}{Community\+:\+:calc\+Fragments} (string fragment\+\_\+file)]Only rectangular fragments will be detected. Problems will also be encountered for adjacent fragments.  
\item[\label{bug__bug000001}%
\hypertarget{bug__bug000001}{}%
Member \hyperlink{class_community_a579c5f423fc2461838a80baf6b396310}{Community\+:\+:detect\+Dimensions} (string db)]If species do not exist across the whole range of the samplemask, samplemask size will not be set correctly and samplemask referencing may be incorrect.  
\item[\label{bug__bug000003}%
\hypertarget{bug__bug000003}{}%
Member \hyperlink{class_spatial_tree_ae97336318c81e182e9f445f7efdbff8d}{Spatial\+Tree\+:\+:setup} () override]For values of dispersal, forest transform rate and time since pristine (and any other double values), they will not be correctly outputted to the S\+I\+M\+U\+L\+A\+T\+I\+O\+N\+\_\+\+P\+A\+R\+A\+M\+E\+T\+E\+RS table if the value is smaller than 10e-\/6. The solution is to implement string output mechanisms using boost\+::lexical\+\_\+cast(), but this has so far only been deemed necessary for the speciation rate (which is intrinsically very small).
\end{DoxyRefList}